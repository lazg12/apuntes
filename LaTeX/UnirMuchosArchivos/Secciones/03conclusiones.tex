
\section{Texto sobre una figura}

ggfdg gdfg gd gfdg gfgsgf ggsfg ggsdfg jhgjghjhk kjhgkhgkhg khgkhgkgh kjhkhkggkh khgkhgkh vxcvx cvxcv xcbcbvc bvcbv cbcvvnv nbmbn mnbmn bmn bmnb nbmnbmnb mn mnb mnbmbnmnb.
ggfdg gdfg gd gfdg gfgsgf ggsfg ggsdfg jhgjghjhk kjhgkhgkhg khgkhgkgh kjhkhkggkh khgkhgkh vxcvx cvxcv xcbcbvc bvcbv cbcvvnv nbmbn mnbmn bmn bmnb nbmnbmnb mn mnb mnbmbnmnb.

\begin{figure}[h]
	\color{colorGris1}\centering
	\begin{overpic}[width=0.6\linewidth, tics=5, grid]{plot}
		% Borrar grilla se quita el comando grid 
		\put (20,1) {5}
		\put (40,1) {10}
		\put (60,1) {15}
		\put (80,1) {20}
		\put (0,20) {5}
		\put (0,40) {10}
		\put (0,60) {15}
		\put (63,67) {$ \boxed{f(x) = x^3} $}
	\end{overpic}
	\caption{Gráfico de una función}
	\label{fig:funcion1}
\end{figure}

ggfdg gdfg gd gfdg gfgsgf ggsfg ggsdfg jhgjghjhk kjhgkhgkhg khgkhgkgh kjhkhkggkh khgkhgkh vxcvx cvxcv xcbcbvc bvcbv cbcvvnv nbmbn mnbmn bmn bmnb nbmnbmnb mn mnb mnbmbnmnb.
ggfdg gdfg gd gfdg gfgsgf ggsfg ggsdfg jhgjghjhk kjhgkhgkhg khgkhgkgh kjhkhkggkh khgkhgkh vxcvx cvxcv xcbcbvc bvcbv cbcvvnv nbmbn mnbmn bmn bmnb nbmnbmnb mn mnb mnbmbnmnb.
ggfdg gdfg gd gfdg gfgsgf ggsfg ggsdfg jhgjghjhk kjhgkhgkhg khgkhgkgh kjhkhkggkh khgkhgkh vxcvx cvxcv xcbcbvc bvcbv cbcvvnv nbmbn mnbmn bmn bmnb nbmnbmnb mn mnb mnbmbnmnb.
ggfdg gdfg gd gfdg gfgsgf ggsfg ggsdfg jhgjghjhk kjhgkhgkhg khgkhgkgh kjhkhkggkh khgkhgkh vxcvx cvxcv xcbcbvc bvcbv cbcvvnv nbmbn mnbmn bmn bmnb nbmnbmnb mn mnb mnbmbnmnb. \footnote{\textcolor{colorAzul1}Justo está bueno poner una nota al pie de página.}

\section{El entorno table}

\begin{table}[h]
	\centering
	\caption{Ejemplo de tabla.}
	\begin{tabular}{|c|c|c|c|}
		\hline
		& 1 & 2 & 3 \\
		\hline
		A & & & \\
		\hline
		B & & & \\
		\hline
	\end{tabular}
	\label{tab:datos}
\end{table}
Esta es la referencia a la tabla \ref{tab:datos}

\section{Personalizar tablas}

ggfdg gdfg gd gfdg gfgsgf ggsfg ggsdfg jhgjghjhk kjhgkhgkhg khgkhgkgh kjhkhkggkh khgkhgkh vxcvx cvxcv xcbcbvc bvcbv cbcvvnv nbmbn mnbmn bmn bmnb nbmnbmnb mn mnb mnbmbnmnb.

% Tabla sin personalizar
\begin{table}[h]
	\centering
	\caption{Coeficientes parciales de seguridad en ELS.}
	\begin{tabular}{p{4cm}cc}
		\hline \textbf{Tipo de acción} & \textbf{Efecto desfavorable} & \textbf{Efecto favorable} \\
		\hline Permanente & gG=1 & gG=1 \\
		\hline Pretensado & 1.10 & 0.90 \\
		\hline Permanente de valor no constante & 1.00 & 1.00 \\
		\hline Variable &.00 & 0.00 \\
		\hline
	\end{tabular}
	\label{tab:coeficientes1}
\end{table}

Esta es la referencia a la tabla \ref{tab:datos}

% Tabla sin personalizada
\begin{table}[h]
	\centering
	\caption{Coeficientes parciales de seguridad en ELS.}
	\rowcolors{1}{white}{gray}
	\arrayrulecolor{colorAzul1}
	{\color{colorAzul1}
		\begin{tabular}{p{4cm}cc}
			\toprule % Primera línea
			\hline \textbf{Tipo de acción} & \textbf{Efecto desfavorable} & \textbf{Efecto favorable} \\
			\midrule % Segunda línea
			Permanente & $ \gamma_G= $1.00 & $ \gamma_G= $1.00 \\
			Pretensado & 1.10 & 0.90 \\
			Permanente de valor no constante & 1.00 & 1.00 \\
			Variable &.00 & 0.00 \\
			\bottomrule % Última línea
	\end{tabular}}
	\label{tab:coeficientes2}
\end{table}

Esta es la referencia a la tabla \ref{tab:datos}


sfsdafa sagdgdf ggfdg gdfg gd gfdg gfgsgf ggsfg ggsdfg jhgjghjhk kjhgkhgkhg khgkhgkgh kjhkhkggkh khgkhgkh vxcvx cvxcv xcbcbvc bvcbv cbcvvnv nbmbn mnbmn bmn bmnb nbmnbmnb mn mnb mnbmbnmnb.

\section{Excel2LaTeX}

ggfdg gdfg gd gfdg gfgsgf ggsfg ggsdfg jhgjghjhk kjhgkhgkhg khgkhgkgh kjhkhkggkh khgkhgkh vxcvx cvxcv xcbcbvc bvcbv cbcvvnv nbmbn mnbmn bmn bmnb nbmnbmnb mn mnb mnbmbnmnb.

% Table generated by Excel2LaTeX from sheet 'Sheet1'
\begin{table}[htbp]
	\centering
	\caption{Tablas de excel}
	\begin{tabular}{p{2.39em}p{15.28em}lllr}
		\multicolumn{1}{l}{} & \multicolumn{1}{r}{} & \multicolumn{4}{c}{\textbf{2021}} \\
		\midrule
		\textbf{Etapa} & \textbf{Productos/Entregables} & \multicolumn{1}{p{4.055em}}{\textbf{Ene-Mar}} & \multicolumn{1}{p{4.055em}}{\textbf{Abr-Jun}} & \multicolumn{1}{p{4.055em}}{\textbf{Jul-Set}} & \multicolumn{1}{p{4.055em}}{\textbf{Oct-Dic}} \\
		\midrule
		E1    & Primer documento a entregar, en .doc. &       & \cellcolor[rgb]{ .718,  .871,  .91} &       &  \\
		E2    & Segundo documento, más pesado. & \cellcolor[rgb]{ .776,  .89,  .859} &       &       &  \\
		E2    & Este tema mejor que no falte. &       & \cellcolor[rgb]{ .776,  .89,  .859} &       &  \\
		E2    & Otro tema interesante en PDF. &       &       & \cellcolor[rgb]{ .776,  .89,  .859} &  \\
		E2    & Esto es importante, también va. &       &       & \cellcolor[rgb]{ .776,  .89,  .859} &  \\
		E3    & Un documento para ver como vamos, editable. &       & \cellcolor[rgb]{ .722,  .804,  .894} &       &  \\
		E4    & Informe de evaluación hasta ahora. &       & \cellcolor[rgb]{ .722,  .804,  .894} &       &  \\
		E5    & Diseño final del sistema. &       &       &       & \cellcolor[rgb]{ .722,  .804,  .894} \\
		E6    & Recomendaciones estratégicas. &       &       &       & \cellcolor[rgb]{ .722,  .804,  .894} \\
		\bottomrule
	\end{tabular}%
	\label{tab:cronograma}%
\end{table}%
final de la tabla gdfg gd gfdg gfgsgf ggsfg ggsdfg jhgjghjhk kjhgkhgkhg khgkhgkgh kjhkhkggkh khgkhgkh vxcvx cvxcv xcbcbvc bvcbv cbcvvnv nbmbn mnbmn bmn bmnb nbmnbmnb mn mnb mnbmbnmnb.

\newpage
\section{Referencias cruzadas}
Esta es la referencia a la \autoref{tab:cronograma} en la \autopageref{tab:cronograma}.

ggfdg gdfg gd gfdg gfgsgf ggsfg ggsdfg jhgjghjhk kjhgkhgkhg khgkhgkgh kjhkhkggkh khgkhgkh vxcvx cvxcv xcbcbvc bvcbv cbcvvnv nbmbn mnbmn bmn bmnb nbmnbmnb mn mnb mnbmbnmnb.

ggfdg gdfg gd gfdg gfgsgf ggsfg ggsdfg jhgjghjhk kjhgkhgkhg khgkhgkgh kjhkhkggkh khgkhgkh vxcvx cvxcv xcbcbvc bvcbv cbcvvnv nbmbn mnbmn bmn bmnb nbmnbmnb mn mnb mnbmbnmnb.

\section{Mini páginas}
\begin{center}
	\begin{minipage}{0.47\linewidth} % Para que puedan entrar dos minipaginas en columnas
		\includegraphics[width=1\linewidth]{example-image-a}
	\end{minipage}
	\begin{minipage}{0.47\linewidth}
		\includegraphics[width=1\linewidth]{example-image-b}
	\end{minipage}
\end{center}

\begin{center}
	\begin{minipage}{0.47\linewidth} % Para que puedan entrar dos minipaginas en columnas
		\includegraphics[width=1\linewidth]{example-image-a}
		\captionof{figure}{Leyenda de la figura A.}
	\end{minipage}\hfil
	\begin{minipage}{0.3\linewidth}
		ggfdg gdfg gd gfdg gfgsgf ggsfg ggsdfg jhgjghjhk kjhgkhgkhg khgkhgkgh kjhkhkggkh khgkhgkh vxcvx cvxcv xcbcbvc bvcbv cbcvvnv nbmbn mnbmn bmn bmnb nbmnbmnb mn mnb mnbmbnmnb
	\end{minipage}
\end{center}

\section{Entorno matemático}
ggfdg gdfg gd gfdg gfgsgf ggsfg ggsdfg jhgjghjhk kjhgkhgkhg khgkhgkgh kjhkhkggkh khgkhgkh vxcvx cvxcv xcbcbvc bvcbv cbcvvnv nbmbn mnbmn bmn bmnb nbmnbmnb mn mnb mnbmbnmnb. La ecuación $ E = m \cdot c^2 $.

Tenemos también $ f_{yk} $ o sino $ x_{a}^{2} $, también tenemos $ \frac{2}{4} $ o sino $ \dfrac{3}{4} $ o la raíz $-\sqrt{2}$, centrar la siguiente ecuación el línea aparte y fuera del párrafo: $$E = m \cdot c^2 $$

La ecuación \ref{eq:energía} se referencia solo el número, pero podemos escribir todo así \autoref{eq:energía} o sino la utilizada en los libros que es así: \eqref{eq:energía}

Numerar ecuaciones en su propio entorno:
\begin{equation} \label{eq:energía}
	E = m \cdot c^2
\end{equation}

Ecuación sin numerar dentro de un entorno
\begin{equation*}
	E = m \cdot c^2
\end{equation*}


\section{Símbolos}
La suma es $ a + b $, la resta es $ a - b $, la multiplicación es $ a \cdot b$ o sino $ a \times b $

Para funciones trigonométricas $ \cos(30) $ o sino $ \sin(30) $

Para integrales $$ \int_0^L x^2, \quad \quad 10 < 15 \quad \Rightarrow \quad 15 \geq 10 $$ % El comando \quad es para dar un espacio en blanco pero se puede utilizar también el comando \hspace{4} donde el 4 es la cantidad de espacios.

\section{Matrices y arreglos}

Ecuaciones alineadas y numeradas
\begin{align}
	f(x) & = x^2 + x^3 \\
	& = x^2 + x^2 \cdot x
\end{align}

Ecuaciones alineadas y no numeradas
\begin{align*}
	f(x) & = x^2 + x^3 \\
	& = x^2 + x^2 \cdot x
\end{align*}

Matrices numeradas
\begin{equation}
	A = 
	\begin{pmatrix}
		1 & 2 & 3 \\
		4 & 5 & 6 \\
		7 & 8 & 9
	\end{pmatrix}
\end{equation}

Matrices no numeradas
\begin{equation*}
	A = 
	\begin{pmatrix}
		1 & 2 & 3 \\
		4 & 5 & 6 \\
		7 & 8 & 9
	\end{pmatrix}, \quad
	B = 
	\begin{bmatrix}
		1 & 2 & 3 \\
		4 & 5 & 6 \\
		7 & 8 & 9
	\end{bmatrix}, \quad
	C = 
	\begin{matrix}
		1 & 2 & 3 \\
		4 & 5 & 6 \\
		7 & 8 & 9
	\end{matrix}, \quad
	D = 
	\begin{Bmatrix}
		1 & a & 3 \\
		4 & b_2 & 6 \\
		7 & 8 & 9
	\end{Bmatrix}, \quad
	E = 
	\begin{vmatrix}
		1 & 2 & 3 \\
		4 & 5 & 6 \\
		7 & 8 & 9
	\end{vmatrix}, \quad
	F = 
	\begin{Vmatrix}
		1 & 2 & 3 \\
		4 & 5 & 6 \\
		7 & 8 & 9
	\end{Vmatrix}
\end{equation*} % No tiene que quedar la línea en blanco o sino sale error

\section{Dibujar ecuaciones con la mano}
Ir a asistentes y después asistente de matemáticas ... que está al final del menú.
No olvidar poner el entorno equation para escribir en lenguaje matemático.
\begin{equation}
	\sum _{x=0}^{n}f{\left(x\right)}^{2}+1=0
\end{equation}

\section{Tipos de errores y alertas}


Undefined control sequence.- No conoce el comando. Está mal escrito o no está cargado el paquete necesario.

Missing \$ inserted.- Falta el entorno matemático.

Extra \} or forggotten \$.- Falta cerrar una viñeta o entorno matemático.

File ended while scannig. . ..- Algo empezó con { pero nunca se cerró.
	
	llegal unit of measure.- No se indicó una unidad de medida o distancia o no existe.
	
	File ‘nombre’ not found.- No se encuentra un archivo. No existe o está mal escrito.
	
	Extra alignment tab has been changed to $\backslash$cr.- Faltan o sobran columnas en una tabla.
	
	Environment ‘nombre’ undefined.- No existe o se escribió mal un entorno.\newline
	
	ADVERTENCIAS
	
	No position in optional float specifier.- No se indició posición en figura o tabla [hbtp!].
	
	Reference ‘nombre’ undefined.- No se encuentra una referencia. No existe o está mal escrita.
	
	There were multiply-defined labels.- Cuando tenemos más de un elemento con la misma etiqueta.
	
	\section{Estructura de carpetas}
	1. Bibliografía
	2. Figuras
	3. Secciones
	4. Settings o ajustes
	
	\section{Unir muchos archivos}
	
	\begin{itemize}
		\item paquetes.tex
		\item inicial.tex
		\item 00\_portada.tex
		\item 01\_introducción.tex
		\item 02\_metodología.tex
		\item 03\_conclusiones.tex
	\end{itemize}
	
	
	
	
