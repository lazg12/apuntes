% Paquete de idioma
\usepackage[spanish, es-nodecimaldot, mexico]{babel} % mexico es para que aparezca en laa leyenda de tablas y no cuadros en las tablas.

\usepackage[utf8]{inputenc}

%Fecha
\usepackage[useregional]{datetime2}

% Configuración de página
\usepackage{geometry}
\geometry{left = 15mm, right = 15mm, top=35mm, bottom=20mm, headheight=30mm} % showframe muestra los marcos de las páginas

% Paquete para justificar el texto
\usepackage{ragged2e}

% Referencias
\usepackage{hyperref} % Para hacer hipervínculos
\hypersetup{colorlinks=true, linkcolor=colorAzul1, urlcolor=colorAzul1, linktocpage, hyperfootnotes=true} % Para que aprezca de color rojo por defecto los hipervínculos al poner linkcolor=blue cambia a color azul las referencias

% Paquetes de fuentes
\usepackage{courierten}
\renewcommand*\familydefault{\ttdefault} %% Only if the base font of the document is to be typewriter style
\usepackage[T1]{fontenc}

% Paquetes de colores
\usepackage[table]{xcolor} % Table es para utilizar los colores en tablas también.
\definecolor{colorRojo1}{rgb}{1.0, 0.03, 0.0}
\definecolor{colorAzul1}{rgb}{0.0, 0.75, 1.0}
\definecolor{colorGris1}{rgb}{0.75, 0.75, 0.75}
\definecolor{colorGris2}{rgb}{0.66, 0.66, 0.66}
\definecolor{colorGris3}{RGB}{50, 50, 50}

% Encabezados
\usepackage{lastpage}
\usepackage{fancyhdr}
\pagestyle{fancy}

\color{colorGris1}\renewcommand{\headrulewidth}{0.5mm}
\let\oldheadrule\headrule % Pone como comando antiguo
\renewcommand{\headrule}{\color{colorAzul1}\oldheadrule} % Redefine el comando anterior en sí está sobreescribiendo el comando anterior del encabezado
\renewcommand{\footrulewidth}{0.2mm}
\let\oldfootrule\footrule % Pone como comando antiguo
\renewcommand{\footrule}{\color{colorAzul1}\oldfootrule} % Redefine el comando anterior en sí está sobreescribiendo el comando anterior del pie de página

% Pie de página
\lfoot{\textcolor{colorGris2} {\small \titulo}}
\cfoot{}
\rfoot{\textcolor{colorGris2}{\small Página. \thepage\ / \pageref*{LastPage}}} % El asterisco en pageref es para que no se genere el hipervínculo y tenga otro color}

% Encabezado
\lhead{\includegraphics[width=0.15\textwidth]{logox}}
\chead{\autor}
\rhead{Domingo \DTMsetstyle{ddmmyyyy}\fechainicio}
\usepackage{setspace} % Permite separar entre línea de texto con mucha soltura.
\usepackage{titlesec} % Para editar las secciones
\titleformat{\section}{\color{colorAzul1}\normalfont\Large\bfseries}{\thesection}{1em}{}
\titleformat{\subsection}{\color{colorAzul1}\normalfont\large\bfseries}{\thesubsection}{1em}{}

% Paquetes de figuras
\usepackage{graphicx} % Carga las figuras
\graphicspath{{./Figuras}} % Indica la dirección de las figuras
\usepackage{float} % Permite poner la posición H
\usepackage{subcaption} % Permite poner subfiguras
\usepackage{wrapfig} % Para envolver el texto alrededor de la figura
\usepackage{overpic} % Para poner texto sobre las figuras
\usepackage{caption} % Para las mini páginas poner etiquetas
\usepackage{tikz} % Para cargar el fondo de la portada
\usepackage[font={color = colorAzul1}]{caption} % Para el color de los captions

% Paquete de tablas
\usepackage{booktabs} % Para darle varios tipos de líneas en las tablas
\usepackage{colortbl} % Para darle color a las tablas


% Paquete para ecuaciones matemáticas
\usepackage{mathtools}
